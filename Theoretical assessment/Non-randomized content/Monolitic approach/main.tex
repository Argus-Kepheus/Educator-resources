\documentclass[11pt]{article}
% ======================================================= %
% EXAM METADATA
% ======================================================= %
\newcommand{\subject}{Latin multi-lingual assessment template}
\newcommand{\teacher}{Calvimontes S. C.}
\newcommand{\examdate}{<test day>}
\newcommand{\period}{<year>}
\newcommand{\examversion}{A}
\newcommand{\examduration}{2 hours}
% ======================================================= %
% PACKAGES 
% ======================================================= %
% ENCODING AND LANGUAGE PACKAGES
\usepackage[T1]{fontenc}        % Font encoding
\usepackage[	portuguese,
				spanish,
				french,
				italian,
				german,
				romanian,
				english]{babel} % Multiple language support
% MATH PACKAGES
\usepackage{amsmath}            % Advanced mathematics
\usepackage{amssymb}            % Mathematical symbols
% GENERAL PACKAGES
\usepackage{adjustbox}          % Box adjustment
\usepackage{csquotes}           % Contextual quotes
\usepackage{enumitem}           % Customized lists
\usepackage[
top			=	1.5cm,
bottom		=	1.5cm,
left		=	1cm,
right		=	1cm,
headheight	=	2cm,  		% header space
headsep		=	0.5cm   	% separation between header and text
]{geometry}
\usepackage{graphicx}           % Image inclusion
\usepackage{ifthen}             % Conditional structures
\usepackage{microtype}          % Typography improvement
\usepackage{multicol}           % Multiple columns
\usepackage{ragged2e}           % Text alignment
\usepackage{tikz}               % Graphics and drawings
\usepackage{titlesec}           % Title formatting
\usepackage{xcolor}             % Colors
% LAYOUT PACKAGES
\usepackage{fancyhdr}
\usepackage{lastpage}
% HYPERLINK PACKAGE (should come last)
\usepackage{hyperref}           % Links and references
% ======================================================= %
% GRADE BOX COMMAND
% ======================================================= %
\newcommand{\gradebox}{%
	\begin{tikzpicture}[remember picture,overlay]
		\node[anchor=north east, xshift=-1.0cm, yshift=-1.0cm] at (current page.north east) {%
			\fbox{\parbox[c][2cm][c]{2cm}{\phantom{.}}}%
		};
	\end{tikzpicture}%
}
% ======================================================= %
% HEADER AND FOOTER CONFIGURATION
% ======================================================= %
\pagestyle{fancy}
\fancyhf{} % Clear all headers and footers
% Header configuration
\fancyhead[L]{\leftmark} % Section name on the left side
\fancyhead[R]{} % Right side empty (or can add other info)
% Footer configuration
\fancyfoot[C]{\thepage\ of \pageref{LastPage}} % Current page of total in center
% Line configuration
\renewcommand{\headrulewidth}{0.4pt} % Header line
\renewcommand{\footrulewidth}{0.4pt} % Footer line
% ======================================================= %
% TYPOGRAPHY SETTINGS
% ======================================================= %
\tolerance=1000
\hyphenpenalty=1000
\emergencystretch=\maxdimen
% Hyphenation settings
\lefthyphenmin=2
\righthyphenmin=3
% ======================================================= %
% ANSWER KEY SETTINGS
% ======================================================= %
\newboolean{showanswers}
\setboolean{showanswers}{true} % true for answer key, false for students
% ======================================================= %
% HYPERREF SETTINGS
% ======================================================= %
\hypersetup{
	colorlinks		=	false,
	pdfborder		=	{0 0 0},
	pdftitle		=	{\subject - Exam},
	pdfsubject		=	{Assessment},
	pdfkeywords		=	{exam, \subject}
}
% ======================================================= %
% IMAGE WITH FALLBACK COMMAND
% ======================================================= %
\newcommand{\imagewithfallback}[3][]{%
	\IfFileExists{figures/#2}{%
		\includegraphics[#1]{figures/#2}%
	}{%
		\fbox{%
			\parbox{0.4\textwidth}{%
				\centering%
				\textcolor{red}{\textbf{IMAGE NOT FOUND}}\\[0.5em]%
				\textcolor{gray}{\footnotesize #3}\\[0.5em]%
				\textcolor{gray}{\footnotesize File: figures/#2}%
			}%
		}%
	}%
}
% ======================================================= %
% LAYOUT SETTINGS
% ======================================================= %
\newlength{\columnspacing}
\setlength{\columnspacing}{0.8cm}
\newcommand{\questionfontsize}{\normalsize}
\newlength{\alternativespacing}
\setlength{\alternativespacing}{0.5em}
\newlength{\questionspacing}
\setlength{\questionspacing}{-0.5em}
\newlength{\questiontoalternativespacing}
\setlength{\questiontoalternativespacing}{0.5em}
% ======================================================= %
% ALTERNATIVE COMMANDS
% ======================================================= %
\newcommand{\Alternative}[2]{%
	\vspace{\questiontoalternativespacing}%
	\noindent%
	\ifthenelse{\boolean{showanswers}}{%
		\ifthenelse{\equal{#2}{}}{%
			\ensuremath{\square}% incorrect alternative
		}{%
			\ensuremath{\blacksquare}% correct alternative
		}%
	}{%
		\ensuremath{\square}% student version
	}%
	\hspace{0.3em}%
	\begin{minipage}[t]{0.88\linewidth}
		\raggedright #1
	\end{minipage}\\[\alternativespacing]
	\setlength{\questiontoalternativespacing}{0pt}%
}
% ======================================================= %
% QUESTION COUNTER AND COMMAND
% ======================================================= %
% Counter for automatic question numbering
\newcounter{questioncounter}
\setcounter{questioncounter}{0}
% Main command for questions with automatic numbering (SAME LINE)
\newcommand{\Question}[1][]{%
	\setlength{\questiontoalternativespacing}{0.5em}% Restore spacing
	\stepcounter{questioncounter}% Increment counter
	\vspace{\questionspacing}% Spacing before question
	\noindent% Remove indentation
	\ifthenelse{\equal{#1}{}}{%
		\textbf{\questionfontsize Question \thequestioncounter:} % Automatic numbering
	}{%
		\textbf{\questionfontsize #1} % Custom text if provided
	}%
	\setlength{\parindent}{0pt}% Remove paragraph indentation
}
% Optional command to reset counter (useful between sections)
\newcommand{\resetquestions}{\setcounter{questioncounter}{0}}
% Optional command to set a specific number for the next question
\newcommand{\setquestion}[1]{\setcounter{questioncounter}{#1}\addtocounter{questioncounter}{-1}}

% Legacy command support (for backwards compatibility)
\newcommand{\pergunta}[1][]{\Question[#1]}
\newcommand{\alternativa}[2]{\Alternative{#1}{#2}}

% ======================================================= %
% SECTIONS
% ======================================================= %
\newcommand{\sectiontitle}[1]{%
	\phantomsection%
	\addcontentsline{toc}{section}{#1}%
	\markboth{#1}{#1}% Sets the section header
	\noindent
	\adjustbox{valign=c,width=\linewidth}{%
		\colorbox{gray!20}{%
			\parbox{\linewidth}{%
				\centering%
				\large\bfseries #1%
			}%
		}%
	}%
	\par
	\vspace{1em}%
}
% ======================================================= %
% COLUMN BREAK COMMAND
% ======================================================= %
\newcommand{\breakcolumnhere}{\vfill\columnbreak}
% ======================================================= %
% STANDARDIZED HEADER
% ======================================================= %
\newcommand{\examheader}{%
	\gradebox
	\begin{center}
		\vspace{-1cm}
		\imagewithfallback[width=0.4\textwidth]{institution-logomark.png}{<institution name>}\\[-0.3em]
		
		{\scriptsize\color{blue!50!black}
			<institution address>\\
			<institution contact info>
		} \vspace{1.5em}
		
		\adjustbox{valign=c}{%
			\colorbox{gray!20}{%
				\parbox[c]{\linewidth}{%
					\centering%
					\textbf{\Large \subject}%
				}%
			}%
		}
		\vspace{-1.0em}
	\end{center}
	
	\vspace{0.3em}
	\textbf{Full name:} \rule{0.4\linewidth}{0.4pt} \hfill
	\textbf{ID number:} \rule{0.25\linewidth}{0.4pt}
	
	\vspace{0.5em}
	{\small
		\textbf{Date:} \examdate \hfill
		\textbf{Period:} \period \hfill
		\textbf{Version:} \examversion \hfill
		\textbf{Duration:} \examduration \hfill
		\textbf{Teacher:} \teacher
	}
	
	\vspace{1em}
	\textbf{Instructions:}
	\begin{itemize}[leftmargin=4.0em, itemsep=-0.2em]
		\item Read questions and answer options carefully before making your choice.
		\item Mark the correct answer by completely filling the square of the chosen option.
		\item Each question has only one correct answer. Make sure to mark only one option.
		\item Use blue or black pen. Do not use pencil or red pen.
	\end{itemize}
	
	\vspace{0.5em}
	\hrule
	\vspace{0.5em}
	\thispagestyle{firstpage}%
}
% ======================================================= %
% SPECIAL CONFIGURATION FOR FIRST PAGE
% ======================================================= %
\fancypagestyle{firstpage}{%
	\fancyhf{}%
	\fancyfoot[C]{\thepage\ of \pageref{LastPage}}%
	\renewcommand{\headrulewidth}{0pt}%
	\renewcommand{\footrulewidth}{0.4pt}%
}
% ======================================================= %
\begin{document}
	\thispagestyle{firstpage}
	\examheader
	
	\setlength{\columnsep}{\columnspacing}
	\setlength{\columnseprule}{0pt}
	\begin{multicols}{2}
		
	\selectlanguage{portuguese}
	\sectiontitle{Hino Nacional Brasileiro}
	
	\Question
	Considerando o processo de oficialização do Hino Nacional Brasileiro no final do século XIX, qual compositor foi responsável pela criação da melodia que se tornou oficialmente reconhecida em 1890?
	
	\Alternative{Carlos Gomes}{}
	\Alternative{Alberto Nepomuceno}{}
	\Alternative{Francisco Manuel da Silva}{OK}
	\Alternative{Heitor Villa-Lobos}{}
	\Alternative{Antônio Carlos Jobim}{}
	
	\Question
	O Hino Nacional Brasileiro passou por diferentes fases. Em que data ele foi executado pela primeira vez, ainda sob o título de \enquote{Hino ao 7 de abril}, em homenagem à abdicação de Dom Pedro I?
	
	\Alternative{7 de setembro de 1822}{}
	\Alternative{15 de novembro de 1889}{}
	\Alternative{21 de abril de 1831}{}
	\Alternative{25 de março de 1824}{}
	\Alternative{13 de abril de 1831}{OK}
	
	\Question
	Durante as comemorações do centenário da Independência do Brasil, foi composta a letra que acompanha o Hino Nacional Brasileiro até hoje. Quem foi o autor que se tornou parte oficial do hino?
	
	\Alternative{Olavo Bilac}{}
	\Alternative{Gonçalves Dias}{}
	\Alternative{Castro Alves}{}
	\Alternative{Joaquim Osório Duque-Estrada}{OK}
	\Alternative{Manuel Bandeira}{}
	
	\Question
	De acordo com a Lei n° 5.700/1971, que regulamenta os símbolos nacionais, em qual das situações abaixo é exigida a execução apenas da introdução e do coro do Hino Nacional Brasileiro, sem a repetição da segunda parte?
	
	\Alternative{Cerimônias religiosas de sentido patriótico}{OK}
	\Alternative{Abertura de sessões cívicas}{}
	\Alternative{Hasteamento da Bandeira Nacional}{}
	\Alternative{Cerimônias oficiais do Governo Federal}{}
	\Alternative{Programas de rádio e televisão às 19 horas}{}
	
	\Question
	O Hino Nacional Brasileiro contém diversas imagens poéticas que exaltam o país. Qual dos versos abaixo faz uma referência direta à constelação do Cruzeiro do Sul, símbolo presente na bandeira nacional?
	
	\Alternative{\enquote{Gigante pela própria natureza}}{}
	\Alternative{\enquote{Fulguras, ó Brasil, florão da América}}{}
	\Alternative{\enquote{A imagem do Cruzeiro resplandece}}{OK}
	\Alternative{\enquote{Em teu formoso céu, risonho e límpido}}{}
	\Alternative{\enquote{Do que a terra mais garrida}}{}
	
	% Spanish section
	\selectlanguage{spanish}
	\sectiontitle{Himno Nacional de Bolivia}
	
	\Question
	En el contexto de la historia musical boliviana, ?`quién fue el compositor responsable de la música del Himno Nacional de Bolivia, pieza que ha acompañado actos oficiales desde el siglo XIX?
	
	\Alternative{José Ignacio de Sanjinés}{}
	\Alternative{Eduardo Abaroa}{}
	\Alternative{Leopoldo Benedetto Vincenti}{OK}
	\Alternative{Franz Tamayo}{}
	\Alternative{Modesto Omiste}{}
	
	\Question
	La letra del Himno Nacional de Bolivia fue escrita por un destacado intelectual boliviano, reflejando el espíritu patriótico de la época. ?`Quién fue el autor de esta letra?
	
	\Alternative{Ricardo José Bustamante}{}
	\Alternative{Franz Tamayo}{}
	\Alternative{Adela Zamudio}{}
	\Alternative{Manuel José Cortés}{}
	\Alternative{José Ignacio de Sanjinés}{OK}
	
	\Question
	El Himno Nacional de Bolivia fue oficialmente adoptado en una fecha significativa para la consolidación de los símbolos patrios. ?`En qué año se oficializó su uso como himno nacional?
	
	\Alternative{1825}{}
	\Alternative{1842}{}
	\Alternative{1845}{OK}
	\Alternative{1851}{}
	\Alternative{1879}{}
	
	\Question
	El Himno Nacional de Bolivia comienza con un verso que evoca el destino favorable de la nación. ?`Cuál es el primer verso que da inicio a esta emblemática composición?
	
	\Alternative{\enquote{De la Patria, el alto nombre}}{}
	\Alternative{\enquote{Compatriotas, la Patria muriendo}}{}
	\Alternative{\enquote{Bolivianos, el hado propicio}}{OK}
	\Alternative{\enquote{Gloria a Bolivia, tierra bendita}}{}
	\Alternative{\enquote{Surge ya la nueva aurora}}{}
	
	\Question
	Además del coro, el Himno Nacional de Bolivia está compuesto por varias estrofas que desarrollan su mensaje patriótico. ?`Cuántas estrofas tiene oficialmente esta obra musical?
	
	\Alternative{Tres estrofas}{}
	\Alternative{Cuatro estrofas}{}
	\Alternative{Cinco estrofas}{}
	\Alternative{Seis estrofas}{}
	\Alternative{Nueve estrofas}{OK}
	
	% French section
	\selectlanguage{french}
	\sectiontitle{L'Hymne National Français}
	
	\Question
	La Marseillaise, hymne emblématique de la République française, a été composée en 1792 dans un contexte de guerre révolutionnaire. Quel musicien est à l'origine de cette œuvre patriotique devenue symbole national ?
	
	\Alternative{Claude Joseph Rouget de Lisle}{OK}
	\Alternative{Jean-Baptiste Lully}{}
	\Alternative{Hector Berlioz}{}
	\Alternative{François-Joseph Gossec}{}
	\Alternative{Jean-Philippe Rameau}{}
	
	\Question
	La ville où fut composée la Marseillaise joue un rôle important dans son histoire et son nom. Dans quelle ville française ce chant révolutionnaire a-t-il été créé ?
	
	\Alternative{Paris}{}
	\Alternative{Marseille}{}
	\Alternative{Lyon}{}
	\Alternative{Strasbourg}{OK}
	\Alternative{Bordeaux}{}
	
	\Question
	Avant d’être connue sous le nom de \enquote{La Marseillaise}, cette œuvre portait un autre titre reflétant son usage militaire initial. Quel était le titre original donné à ce chant lors de sa création ?
	
	\Alternative{Chant de guerre de l'armée française}{}
	\Alternative{Hymne des Marseillais}{}
	\Alternative{Chant de guerre pour l'armée du Rhin}{OK}
	\Alternative{Chant patriotique de la République}{}
	\Alternative{Marche des volontaires}{}
	
	\Question
	La reconnaissance officielle d’un hymne national est un acte symbolique fort. En quelle année la Marseillaise a-t-elle été adoptée comme hymne national officiel de la France ?
	
	\Alternative{1792}{}
	\Alternative{1795}{OK}
	\Alternative{1804}{}
	\Alternative{1815}{}
	\Alternative{1848}{}
	
	\Question
	Le nom \enquote{La Marseillaise} ne provient pas de son lieu de composition, mais d’un événement marquant lié à sa diffusion. Pourquoi ce chant porte-t-il ce nom ?
	
	\Alternative{Il a été composé à Marseille}{}
	\Alternative{Il célèbre une victoire à Marseille}{}
	\Alternative{Les fédérés marseillais l'ont popularisé à Paris}{OK}
	\Alternative{Le compositeur était marseillais}{}
	\Alternative{Il a été chanté pour la première fois à Marseille}{}
	
	% Italian section
	\selectlanguage{italian}
	\sectiontitle{L'Inno Nazionale Italiano}
	
	\Question
	L'inno nazionale italiano è conosciuto anche con il titolo \enquote{Fratelli d'Italia}. Chi fu l'autore del testo, scritto in un periodo di forte sentimento patriottico durante il Risorgimento?
	
	\Alternative{Giuseppe Verdi}{}
	\Alternative{Alessandro Manzoni}{}
	\Alternative{Goffredo Mameli}{OK}
	\Alternative{Giuseppe Garibaldi}{}
	\Alternative{Gabriele D'Annunzio}{}
	
	\Question
	La melodia dell'inno nazionale italiano è stata composta da un musicista ligure, su richiesta dell'autore del testo. Chi fu il compositore della musica dell'Inno di Mameli?
	
	\Alternative{Gioachino Rossini}{}
	\Alternative{Michele Novaro}{OK}
	\Alternative{Giuseppe Verdi}{}
	\Alternative{Vincenzo Bellini}{}
	\Alternative{Pietro Mascagni}{}
	
	\Question
	Il testo dell'inno nazionale italiano fu scritto in un momento cruciale del Risorgimento. In quale anno Goffredo Mameli compose le parole de \enquote{Il Canto degli Italiani}?
	
	\Alternative{1846}{}
	\Alternative{1847}{OK}
	\Alternative{1848}{}
	\Alternative{1861}{}
	\Alternative{1870}{}
	
	\Question
	L'inno nazionale italiano inizia con un verso che è diventato simbolo dell'identità nazionale. Quale tra i seguenti versi apre ufficialmente l'inno?
	
	\Alternative{\enquote{Italia, Italia, sacra terra}}{}
	\Alternative{\enquote{Viva l'Italia unita e forte}}{}
	\Alternative{\enquote{Fratelli d'Italia, l'Italia s'è desta}}{OK}
	\Alternative{\enquote{Gloria all'Italia nostra}}{}
	\Alternative{\enquote{Sorgi Italia, dal tuo sonno}}{}
	
	\Question
	Sebbene fosse già ampiamente utilizzato, \enquote{Il Canto degli Italiani} è stato riconosciuto ufficialmente come inno nazionale solo in epoca repubblicana. In quale anno è avvenuta questa ufficializzazione?
	
	\Alternative{1946}{OK}
	\Alternative{1948}{}
	\Alternative{1861}{}
	\Alternative{1922}{}
	\Alternative{1870}{}
	
	% German section
	\selectlanguage{german}
	\sectiontitle{Die Deutsche Nationalhymne}
	
	\Question
	Die Melodie der deutschen Nationalhymne stammt von einem bekannten Komponisten der Klassik. Wer komponierte ursprünglich diese Melodie, die heute mit dem Deutschlandlied verbunden ist?
	
	\Alternative{Ludwig van Beethoven}{}
	\Alternative{Johann Sebastian Bach}{}
	\Alternative{Joseph Haydn}{OK}
	\Alternative{Richard Wagner}{}
	\Alternative{Johannes Brahms}{}
	
	\Question
	Der Text des Deutschlandliedes wurde im 19. Jahrhundert verfasst und spiegelt die damaligen nationalen Ideale wider. Wer war der Autor dieses Liedtextes?
	
	\Alternative{Johann Wolfgang von Goethe}{}
	\Alternative{Friedrich Schiller}{}
	\Alternative{Heinrich Heine}{}
	\Alternative{August Heinrich Hoffmann von Fallersleben}{OK}
	\Alternative{Theodor Körner}{}
	
	\Question
	Das Deutschlandlied besteht aus drei Strophen, doch seit 1952 wird nur eine davon offiziell als Nationalhymne verwendet. Welche Strophe wird seitdem bei offiziellen Anlässen gesungen?
	
	\Alternative{Die erste Strophe}{}
	\Alternative{Die zweite Strophe}{}
	\Alternative{Die dritte Strophe}{OK}
	\Alternative{Alle drei Strophen}{}
	\Alternative{Die erste und dritte Strophe}{}
	
	\Question
	Die heute gesungene Strophe der deutschen Nationalhymne beginnt mit einem markanten Vers, der zentrale Werte der Bundesrepublik Deutschland betont. Mit welchen Worten beginnt diese Strophe?
	
	\Alternative{\enquote{Deutschland, Deutschland über alles}}{}
	\Alternative{\enquote{Deutsche Frauen, deutsche Treue}}{}
	\Alternative{\enquote{Einigkeit und Recht und Freiheit}}{OK}
	\Alternative{\enquote{Blüh im Glanze dieses Glückes}}{}
	\Alternative{\enquote{Von der Maas bis an die Memel}}{}
	
	\Question
	Das Deutschlandlied wurde in einer Zeit des politischen Umbruchs und der nationalen Bewegung verfasst. In welchem Jahr wurde der Text dieses Liedes ursprünglich geschrieben?
	
	\Alternative{1797}{}
	\Alternative{1815}{}
	\Alternative{1841}{OK}
	\Alternative{1871}{}
	\Alternative{1918}{}
	
	% Romanian section
	\selectlanguage{romanian}
	\sectiontitle{Imnul Național al României}
	
	\Question
	Imnul național al României, cunoscut sub titlul \enquote{Deșteaptă-te, române}, are o istorie profund legată de mișcările revoluționare din secolul al XIX-lea. Cine este autorul versurilor acestui imn, care exprimă chemarea la unitate și libertate?
	
	\Alternative{Vasile Alecsandri}{}
	\Alternative{Mihai Eminescu}{}
	\Alternative{Andrei Mureșanu}{OK}
	\Alternative{George Coșbuc}{}
	\Alternative{Ion Heliade Rădulescu}{}
	
	\Question
	Melodia imnului \enquote{Deșteaptă-te, române} a fost compusă pentru a însoți versurile lui Andrei Mureșanu. Cine este compozitorul care a creat această muzică devenită simbol național?
	
	\Alternative{Ciprian Porumbescu}{}
	\Alternative{Anton Pann}{OK}
	\Alternative{Tiberiu Brediceanu}{}
	\Alternative{George Enescu}{}
	\Alternative{Dinu Lipatti}{}
	
	\Question
	Imnul actual al României a fost adoptat oficial într-un moment de schimbare politică majoră. În ce an a devenit \enquote{Deșteaptă-te, române} imnul național al țării?
	
	\Alternative{1918}{}
	\Alternative{1947}{}
	\Alternative{1989}{}
	\Alternative{1990}{OK}
	\Alternative{2001}{}
	
	\Question
	Versurile imnului național românesc exprimă o chemare la trezirea conștiinței naționale. Care este primul vers cu care începe acest imn?
	
	\Alternative{\enquote{Români, uniți-vă în cuget și-n simțiri}}{}
	\Alternative{\enquote{Sculați-vă din somnul cel de moarte}}{}
	\Alternative{\enquote{Deșteaptă-te, române, din somnul cel de moarte}}{OK}
	\Alternative{\enquote{Trăiască România liberă și unită}}{}
	\Alternative{\enquote{Ridică-te, nație mândră și curajoasă}}{}
	
	\Question
	Înainte de adoptarea imnului \enquote{Deșteaptă-te, române}, România a avut mai multe imnuri oficiale. Care dintre următoarele NU a fost niciodată imn național al României?
	
	\Alternative{\enquote{Trăiască Regele}}{}
	\Alternative{\enquote{Pe-al nostru steag e scris Unire}}{}
	\Alternative{\enquote{Trei culori}}{}
	\Alternative{\enquote{Hora Unirii}}{OK}
	\Alternative{\enquote{Deșteaptă-te, române}}{}
	
	% English section
	\selectlanguage{english}
	\sectiontitle{End}

	\end{multicols}	
\end{document}